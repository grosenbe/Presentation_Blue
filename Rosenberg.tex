%----------------------------------------------------------------------------------------
%	PACKAGES AND THEMES
%----------------------------------------------------------------------------------------

\documentclass[aspectratio=169]{beamer}

\mode<presentation> {

\usetheme{Madrid}

%\setbeamertemplate{footline} % To remove the footer line in all slides uncomment this line
%% \setbeamertemplate{footline}[page number] % To replace the footer line in all slides with a simple slide count uncomment this line

\setbeamertemplate{navigation symbols}{} % To remove the navigation symbols from the bottom of all slides uncomment this line
}

\usepackage{graphicx} % Allows including images
\usepackage{booktabs} % Allows the use of \toprule, \midrule and \bottomrule in tables

%% Bunch of stuff to make a flowchart with tikz
\usepackage{tikz}
\usetikzlibrary{shapes.geometric,backgrounds,positioning,
  positioning-plus,node-families,calc}
\tikzset{
  basic box/.style = {
    shape = rectangle,
    align = center,
    draw  = #1,
    fill  = #1!25,
    rounded corners},
  header node/.style = {
    Minimum Width = header nodes,
    font          = \strut\Large\ttfamily,
    text depth    = +0pt,
    fill          = white,
    draw},
  header/.style = {%
    inner ysep = +1.5em,
    append after command = {
      \pgfextra{\let\TikZlastnode\tikzlastnode}
      node [header node] (header-\TikZlastnode) at (\TikZlastnode.north) {#1}
      node [span = (\TikZlastnode)(header-\TikZlastnode)]
        at (fit bounding box) (h-\TikZlastnode) {}
    }
  },
  hv/.style = {to path = {-|(\tikztotarget)\tikztonodes}},
  vh/.style = {to path = {|-(\tikztotarget)\tikztonodes}},
}

%----------------------------------------------------------------------------------------
%	TITLE PAGE
%----------------------------------------------------------------------------------------

\title[Summary Presentation]{Geoff Rosenberg Interview} % The short title appears at the bottom of every slide, the full title is only on the title page

\author{Geoff Rosenberg} % Your name
\institute[] % Your institution as it will appear on the bottom of every slide, may be shorthand to save space
{
%% \medskip
\textit{Geoff.Rosenberg@gmail.com} % Your email address
}
\date{June 1, 2018} % Date, can be changed to a custom date

\begin{document}

\begin{frame}
\titlepage % Print the title page as the first slide
\end{frame}

\begin{frame}
\frametitle{Overview} % Table of contents slide, comment this block out to remove it
\tableofcontents % Throughout your presentation, if you choose to use \section{} and \subsection{} commands, these will automatically be printed on this slide as an overview of your presentation
\end{frame}

%----------------------------------------------------------------------------------------
%	PRESENTATION SLIDES
%----------------------------------------------------------------------------------------

%------------------------------------------------
\section{Personal Summary} % Sections can be created in order to organize your presentation into discrete blocks, all sections and subsections are automatically printed in the table of contents as an overview of the talk
%------------------------------------------------

%% ** First 20 minutes of the presentation should be a high level overview of myself -- call it a personal statement.
%% *** This is where I talk about why I want to go into the space industry
%% **** Overarching mission is I want the world to benefit from the work I do.  That's why I went into renewable energy, and that's why I want to go into the space industry
%% **** Might want to mention that our team won the robotics competition
%% *** Questions I want to answer
%% **** Why did I choose my schools and degrees?
%% ***** Although UW as actually the first school that I heard back from, I chose San Diego for undergrad because it made more financial sense.
%% ***** USC because it has a highly rated MS program for dynamics and controls
%% **** Why am I changing jobs?
%% ***** I've always wanted to work in space

%% \subsection{Subsection Example} % A subsection can be created just before a set of slides with a common theme to further break down your presentation into chunks
\subsection{Goals}
\begin{frame}
  \frametitle{Overarching Mission}
  In essence, I want to go into the space industry it's because I want the world to benefit from the work I do.  That's why I went into renewable energy, and that's why I want to go into the space industry.
\end{frame}

\begin{frame}
  \frametitle{Interests} This is where I'd like to mention that I love
  space travel and \textbf{why}.
  \begin{itemize}
  \item Maybe mention Kerbal Space Program?
  \item Maybe mention that our team in college won the robotics
    competition?  (probably not, no one cares)
  \item Maybe mention building Estes rockets when I was a kid?  And
    demoing them for my 5th grade class.
    \begin{itemize}
    \item I was a cool kid...
    \end{itemize}
  \item I've always loved learning about how things work,
    and solving problems.
  \end{itemize}
\end{frame}

\subsection{Undergrad}
\begin{frame}
  \frametitle{Schools}
  \begin{itemize}
  \item Should mention the schools I went to and why I chose them.
  \item Mention that I almost went to UW.
  \item Break into a couple slides.
  \end{itemize}
\end{frame}

\begin{frame}
  \frametitle{Undergrad (UCSD)}
   % The "c" option specifies centered vertical alignment while the
   % "t" option is used for top vertical alignment
  \begin{columns}[t]
    \column{0.45\textwidth}
  \begin{itemize}
  \item Was and is one of the top engineering programs in the country.
  \item Chose ME because it's interesting (I loved building dams a
    kid), and because it's versatile.
  \item The beach is nice, too :)
  \end{itemize}

  \column{0.65\textwidth}
  \begin{figure}
    \includegraphics[width=0.7\linewidth]{UCSD.jpg}
  \end{figure}
  \end{columns}
\end{frame}

\subsection{Grad school}
\begin{frame}
  \frametitle{Grad School (USC)}
  \begin{columns}[t]
    \column{0.45\textwidth}
  \begin{itemize}
  \item I chose USC because it has a very strong dynamics and controls
    program (my favorite class during undergrad).
  \item I also got to play a lot of music.
  \end{itemize}

  \column{0.65\textwidth}
  \begin{figure}
    \includegraphics[width=0.7\linewidth]{USC.jpg}
  \end{figure}
  \end{columns}
\end{frame}

%------------------------------------------------
\section{Professional Career}
%------------------------------------------------

%% ** Second 30 minutes is an overview of stuff I did at work.
%% *** Mechanical Engineer role (SR)
%% *** Systems engineer role (SR)
%% *** Describe the transition to MFC
%% *** Senior Systems Engineer (MFC)
%% **** Work at the ATB

%TODO transition slide

\subsection{SolarReserve}
\subsubsection{Mechanical Engineer}

\begin{frame}
  %% TODO Find a better picture than this...
  \frametitle{SolarReserve}
  \center
  \includegraphics[width=.7\linewidth]{HeliostatImage.jpg}
\end{frame}

%% Maybe a slide here to describe the state of the company when I joined it?
\begin{frame}
  \frametitle{SolarReserve -- Background}
  \begin{itemize}
  \item SolarReserve was founded in 2008, primarily as power project
    development company rather than a technology development company
    \begin{itemize}
    \item Exclusive worldwide license to Rocketdyne's concentrated
      solar power technology
      \item Approximately tens thousand autonomously tracking solar mirrors
        (called heliostats) collect energy by heating molten salt.
      \item Thermal energy is collected during the day and stored, to
        be dispatched later as needed by the grid.
    \end{itemize}
  \item Joined SolarReserve in early 2011 as employee \#64, as one
    of few technical people at the company
    \begin{itemize}
      \item As the cognizant engineer supporting development of
        photovoltaic power projects in the US, I had to learn quickly
        and be resourceful.
      \end{itemize}
  \end{itemize}
\end{frame}
 
\begin{frame}
  \frametitle{SolarReserve -- As A Mechanical Engineer}
  \begin{itemize}
  \item Developed a photovoltaic power plant performance toolkit.
    \begin{itemize}
    \item A PV performance model and design tool, and the systems
      engineering trades it allowed us to do it quickly.
    \end{itemize}
  \item Worked on the CDSEP plant guarantee model, integrating the
    Rocketdyne and Cobra Fortran code.
  \item Development of the CSP performance model/implementation of the
    smart dispatch logic in Matlab.
  \item Developed and debugged the PV financial model.
    %% \begin{itemize}
    %% \item Although this was a spreadsheet, I ended up learning a bit
    %%   about project finance and internal return calculation.
    %% \end{itemize}
  \end{itemize}
\end{frame}

% PV performance model development slide
% Design of PV plants based on the open circuit voltage, short circuit current of the modules (show a PV IV curve to describe)
\begin{frame}
  \frametitle{SolarReserve -- Photovoltaic Performance Toolkit Development}
  It would be pretty cool to make this kind of a ``web'' layout, rather than just bullets
  \begin{itemize}
  \item Inputs
  \item Outputs
  \item PVSyst or SAM based performance modeling
  \end{itemize}
\end{frame}

\begin{frame}
  \frametitle{PV System Design Tool}
  \begin{columns}[t]
    \column{0.45\textwidth}
    \begin{block}{Inputs}
      \begin{itemize}
      \item Land availability
      \item Module electrical characteristics
      \item Inverter electrical characteristics
      \item Interconnection voltage
      \end{itemize}
    \end{block}

    \column{0.45\textwidth}
    \begin{block}{Outputs}
      \begin{itemize}
      \item Overall Plant DC and AC circuit design
      \item Estimated land required
      \item Plant array capacity [MW$_{dc}$]
      \item Plant interconnection rating [MW$_{ac}$]
      \item PV module simulation software input parameters
      \item Hourly plant output estimation
      \end{itemize}
    \end{block}
  \end{columns}
\end{frame}

%% Describe what knobs we have to turn in designing a plant
%% Generally, the plant's AC capacity is fixed by the interconnection agreement
%% Module and inverter selection
%% Fixed tilt or tracking
%% DC:AC Ratio
%% Ground coverage ratio
%% Inverter MPPT Range
\begin{frame}
\frametitle{Design Criteria for PV Plants}
\begin{table}
\begin{tabular}{l l l}
\toprule
\textbf{Treatments} & \textbf{Response 1} & \textbf{Response 2}\\
\midrule
Treatment 1 & 0.0003262 & 0.562 \\
Treatment 2 & 0.0015681 & 0.910 \\
Treatment 3 & 0.0009271 & 0.296 \\
\bottomrule
\end{tabular}
\end{table}
\end{frame}

\begin{frame}
  \frametitle{DC System Design}
  \begin{block}{Electrical Requirements}
    \begin{columns}
      \column{0.45\textwidth}
      \begin{itemize}
      \item The DC side of a photovoltaic plant block is configured
        with $n$ parallel strings of $m$ modules in series.
      \item Module $I_{max} * n <$ Inverter's DC Bus current rating
        $\rightarrow$ driver for $n$
      \item Module $V_{max} * m <$ Inverter's DC bus voltage rating
        $\rightarrow$ driver for $m$
      \end{itemize}

      \column{0.45\textwidth}
      \begin{figure}
        \textbf{Module Current/Voltage Relationship}
        \includegraphics[width=0.75\linewidth]{IV_Curve.png}
      \end{figure}
    \end{columns}
  \end{block}
  \begin{block}{Land Usage Requirements}
    \begin{itemize}
    \item Trade between packing modules closer together to reduce land
      usage, and causing modules to shade one another at low solar
      elevation in the morning and evening.
    \end{itemize}
  \end{block}
\end{frame}

\begin{frame}
  \frametitle{AC System Design}
  \begin{columns}[t]
    \column{0.65\textwidth}
    \begin{itemize}
    \item Conversion of DC power to AC through PWM.
    \item 60 Hz sinusoid (power grid frequency) broken into multiple
      steady state DC output levels.
    \item Inverter carrier frequency of approximately 4 kHz.
    \item Selection of the minimum DC Bus voltage tracking range
      affects the AC voltage output. (Peak AC voltage = minimum DC
      tracking voltage - haircut)
    \item Design enough inverters into the plant such that there is
      enough AC capacity to cover the total output from the array
      under ideal conditions.
      \begin{itemize}
      \item Energy is wasted otherwise.
      \end{itemize}
    \end{itemize}

    \column{0.45\textwidth}
    \begin{figure}
      \includegraphics[width=0.75\textwidth]{PWM.png}
    \end{figure}    
  \end{columns}
\end{frame}


\begin{frame}
\frametitle{Design Criteria for PV Plants}
How we characterize the ``goodness'' of a particular configuration.
\begin{itemize}
\item Annual Energy Production $\rightarrow$ From performance model.
\item Performance Ratio $\rightarrow$ From performance model.
\item Specific Yield $\rightarrow$ From performance model.
\item Levelized Cost of Energy $\rightarrow$ From financial model.
  \begin{itemize}
  \item All of the above items are inputs for the project financial model.
  \end{itemize}
\end{itemize}
\end{frame}

\begin{frame}
  \frametitle{PV Performance Calculation}
  \begin{itemize}
  \item Matlab code to handle calculation of gen tie losses based on
    power factor assumption, line impedance estimate, and
    interconnection voltage.
  \item $I^{2}Z$ losses.
  \item Optional time of day multiplier; energy delivered during peak
    demand hours is worth more to the customer.
  \item Calculation of P50, P90, and P10 annual outputs, based on
    decades of hourly solar resource data.
  \end{itemize}
\end{frame}

\begin{frame}
  \frametitle{CSP Performance: Flux Model} Development of a Monte
  Carlo ray tracer to estimate the performance of a CSP plant's
  collector field based on the level of solar irradiation
\end{frame}

\subsubsection{Systems Engineer}
%% All of the bullet points below deserve a slide
%TODO do a ``feature set'' drawing for the heliostat characterization to make it clearer how it works
\begin{frame}
  \frametitle{As A Systems Engineer}
  \begin{itemize}
  \item Development of the SR-120 control software (Matlab)
  \item Kalman filter development
  \item Include some stuff about the ``zero'' finding logic?
  \item Development of the SVS-Vistek and ViewWorks camera interfaces
    in C\#
  \item Test campaigns for the SR-120
  \item Determination and definition of performance requirements (IE
    mirror slope and pointing errors)
  \item Hands on testing at Sandia National Labs
  \item Troubleshooting and modifying the AMS software
  \item Camera and SPCA network setup
  \item Daily SCRUMS (agile development) and piloting the system with
    Mark and Roger
  \item All of the performance data processing, which I had to
    automate because the rest of the team started on the SR-96
  \item Development of the in-situ/star characterization software
  \item Development of the control system interface; changing a
    relational object database into an object database.
  \end{itemize}
\end{frame}

%% Describe what the SR-120 is, a few of its features, and how I contributed
\begin{frame}
  \frametitle{SR-120 Heliostat Development}
  \begin{itemize}
  \item Development of a new and disruptive type of heliostat design
  \item Wireless power and communication
  \item Two orders of magnitude smaller than conventional heliostats
    \begin{itemize}
    \item On the order of one million per plant $\rightarrow$ mass
      production becomes a possibility
    \end{itemize}
    \item Optically controlled
  \end{itemize}
\end{frame}

%% Slide on the SR-120 software development
\begin{frame}
  \frametitle{SR-120 Heliostat Development -- Control Software}
  %% Zero finding and Kalman filter development
  %% Background of the Kelly 6 parameter model (Matt Kelly's MIT master's thesis)
    \begin{itemize}
    \item
  \end{itemize}
\end{frame}

\begin{frame}
  \frametitle{SR-120 Heliostat Development -- Hardware Selection and }
  %% Camera Interfaces and AMS software
    \begin{itemize}
    \item
  \end{itemize}
\end{frame}

%% building stuff from Thorlabs components
\begin{frame}
  \frametitle{In-Situ Heliostat Characterization}
  \begin{itemize}
    \item
  \end{itemize}
\end{frame}

%% HabConnect relational database interface to write to the SCADA system's database
\begin{frame}
  \frametitle{In-Situ Characterization -- Software Development}
  \begin{itemize}
  \item
  \end{itemize}
\end{frame}

%% All of the testing campaigns at SNL
\begin{frame}
  \frametitle{Integration and Test -- Sandia}
  \begin{columns}[c]
    \column{0.5\textwidth}
    \begin{figure}
      \textbf{Solar Thermal Test Facility\\Sandia National Labs
        (Albuquerque, NM)}
      \includegraphics[width=\linewidth]{Sandia.png}
    \end{figure}

    \column{0.5\textwidth}
    \begin{itemize}
    \item Field testing of SR-120 hardware and software occurred at
      Sandia National Labs.
    \end{itemize}
  \end{columns}
\end{frame}

\begin{frame}
  \frametitle{Integration and Test -- Crescent Dunes}
  \begin{columns}[c]
    \column{0.5\textwidth}
    \begin{figure}
      \textbf{Crescent Dunes Solar Thermal Power Plant\\
        (Tonopah, NV)}
      \includegraphics[width=\linewidth]{CDSEP.jpg}
    \end{figure}

    \column{0.5\textwidth}
    \begin{itemize}
    \item Description of the in-situ canting field testing
    \end{itemize}
  \end{columns}
\end{frame}

\begin{frame}
  \frametitle{Integration and Test -- Raymer Test Facility}
  \begin{columns}[c]
    \column{0.5\textwidth}
    \begin{figure}
      \textbf{Crescent Dunes Solar Thermal Power Plant\\
        (Tonopah, NV)}
      \includegraphics[width=\linewidth]{CDSEP.jpg}
    \end{figure}

    \column{0.5\textwidth}
    \begin{itemize}
    \item Description of the in-situ canting field testing
    \end{itemize}
  \end{columns}
\end{frame}

\subsection{Lockheed}
% I'm pretty sure the logo is copyrighted, so I shouldn't use it...
\begin{frame}
  \frametitle{Lockheed Martin Missiles and Fire Control}
  \center
  \includegraphics[width=.7\linewidth]{LockheedLogo}
\end{frame}

%reasons for transition
\begin{frame}
  \frametitle{Transitioning from SolarReserve}
  \begin{itemize}
  \item Concerns about R\&D budget cuts.
  \end{itemize}
\end{frame}

\begin{frame}
  \frametitle{Affordability Test Bed}
  \begin{itemize}
  \item Describe what the ATB is and what it's intended to be
  \item Describe the loitering munition simulation and the image delay problem
  \item Software overlay generation/frame numbers
  \item Camera triggering
  \item Describe the oscilloscope testing to determine sync pulse voltage and frequency
  \item Display port $\rightarrow$ DVI $\rightarrow$ HDMI $\rightarrow$ VGA $\rightarrow$ BNC 5 wire (RGB + HSYNC + VSYNC) $\rightarrow$ Alligator clips $\rightarrow$ Camera
  %% \item Show the frame delay histogram ... just generate one myself in Python
  \end{itemize}
\end{frame}

\begin{frame}
  \frametitle{Loitering Munition Simulation}
  %% Describe the loitering munition simulation
  %% Continuous time simulation of a loitering munition
  %% Most of the sim and flight computer software is written in Simulink, and autocoded to c++
  %% All lower level stuff (frame handling, hardware interfaces) is written by hand in c++.  This is what I worked on.
  \begin{itemize}
  \item Continuous time simulation of a loitering munition
    \begin{itemize}
    \item A fixed wing drone with either explosives or EMP on it.
    \end{itemize}
  \end{itemize}
    \begin{figure}
      \includegraphics[width=\textwidth]{Firebird-FPV-Fixed-Wing-Drone.jpg}
    \end{figure}
    
\end{frame}

\begin{frame}
  \frametitle{Loitering Munition Frame Delay Problem}
  \begin{itemize}
  \item Discrepancies between simulation and reality
  \item Non-zero latency between an update of the simulation state (simulation time $t_{\tau}$) and when the image is rendered on the screen ($t_{\tau+\delta}$)
  \end{itemize}
\end{frame}

\begin{frame}
  \frametitle{IR Projector Setup}
  %test IR camera selection
  %sample pixel packing code -- knew how to do this from work at SR on cameras

\end{frame}

\begin{frame}
  \frametitle{PAC-3} % I might want to be a little sparing in my descriptions here, because this stuff is classified
  \begin{itemize}
  \item Intended to engage and destroy tactical ballistic missiles
  \item Most of it is classified
  \item Work on a Linux computing cluster on a continuous time simulation
  \item Integrating changes to the simulation from both internal, and also from the customer (IE the government), as well as the ground system components (which are from Raytheon).
  \item Integration of an older version of the simulation (entirely Fortran/Linux based) and modularizing it into several libraries which the customer can link together.
  \item Builds on Windows or Linux
  \end{itemize}
\end{frame}


\begin{frame}
  \frametitle{Configuration Management on PACNET}
  \begin{itemize}
  \item Handling configuration management on a simulation of about 800,000 SLOC
  \item Handling requests for modification
  \item Installing new releases of our code
  \item Handling updates from the system integrator
  \end{itemize}
\end{frame}

%% Image injection/projection flow charts
%define some colors for the blocks
\definecolor{light_blue}{RGB}{2,127,237}

%% color definitions for the different signals
\definecolor{Ethernet}{RGB}{5,163,0}        %green
\definecolor{HDMI}{RGB}{163,111,0}   %gold
\definecolor{USB}{RGB}{0,0,255}
\definecolor{Data}{RGB}{0,0,0}

\begin{frame}
  \frametitle{Image Projection Flow Chart}
\begin{tikzpicture}[node distance = 0.6cm, thick, nodes = {align = center},
    >=latex]
  %% simulation host
  \node[fill = light_blue] (6_DOF)
       {6DOF state @\\$t_{n}$};
  \node[fill = light_blue, below = of 6_DOF] (Cigi_Cmds)
       {Cigi Update Commands};
  \node[fill = light_blue, below = of Cigi_Cmds] (Frame_Logger)
       {Frame Logger};
  \node[fill = light_blue, below = of Frame_Logger] (PNG)
       {Images Saved as PNG};
  \node[fill = light_blue, right = of PNG] (Timestamp_Log)
       {Logging of time stamps\\and file names};
  \node[fill = light_blue, left = 1.5 of PNG] (Update_Log)
       {Cigi Update Log\\$t_n$, Frame Number};
  \node[fill = light_blue, above = 1.25 of 6_DOF] (Monitor)
       {Monitor};

  %% Image generator
  \node[fill = light_blue, left = 4 of 6_DOF] (Scene_Render)
       {Scene\\Rendering};

  %% Flight computer
  \node[fill = light_blue, right = 1.5 of Cigi_Cmds] (Image)
       {Image Received @\\$t_{n+\delta}$};
  \node[fill = light_blue, above = 1.25 of Image] (Camera)
       {Camera};

  %background nodes to separate out the computers
  \begin{scope}[on background layer]
    \node[fit = (6_DOF)(Cigi_Cmds)(Frame_Logger), basic box = red,
      header = Simulation Host] (Host) {};

    \node[fit = (Scene_Render), basic box = blue,
      header = Image Generator] (Image_Generator) {};

    \node[fit = (Image)(Camera), basic box = blue,
      header = Flight Computer] (Flight_Computer) {};
  \end{scope}

  %% paths to connect the nodes
  \path[very thick, Data] (6_DOF.south) edge[->] (Cigi_Cmds.north);
  \path[very thick, Ethernet] (Cigi_Cmds) edge[->] (Scene_Render);
  \path[very thick, Data] (Cigi_Cmds) edge[->] (Update_Log);
  \path[very thick, HDMI] (Scene_Render) edge[->] (Monitor.west);
  \path[very thick, HDMI] (Monitor.east) edge[->] (Camera.west);
  \path[very thick, USB] (Camera.south) edge[->] (Image);
  \path[very thick, Ethernet] (Image) edge[->] node[pos=0.35, sloped, above] {Image Data} (Frame_Logger);
  \path[very thick, Data] (Frame_Logger) edge[->] (Timestamp_Log);
  \path[very thick, Data] (Frame_Logger) edge[->] (PNG);
\end{tikzpicture}
\end{frame}

\begin{frame}
\Huge{\centerline{Q\&A?}}
\end{frame}

\end{document}
